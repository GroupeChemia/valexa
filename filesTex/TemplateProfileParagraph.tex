\subsubsection{\COMPOUNDname}

\paragraph{Données}
\subparagraph{Calibration}
Le tableau suivant présente les données de calibration pour le composé \COMPOUNDname.

\begin{longtable}{c|c|P{25mm}|P{20mm}}
    \textbf{Série} & \textbf{Niveau} & \textbf{Concentration} & \textbf{Réponse}\\
    \hline\endhead

    \DATAcalibration

\caption{Donnée de calibration pour le composé \COMPOUNDname}
\label{tab:##}
\end{longtable}



\newpage



\subparagraph{Validation}
Le tableau suivant présente les données de validation pour le composé \COMPOUNDname. Les données retrouvées sont celle avec le facteur de correction de XXXXXX.

\begin{longtable}{c|c|P{25mm}|P{20mm}|P{25mm}|P{20mm}|P{20mm}}
    \textbf{Série} & \textbf{Niveau} & \textbf{Concentration} & \textbf{Réponse} & \textbf{Concentration Calculé} & \textbf{Biais Absolu} & \textbf{Biais Relatif}\\
    \hline\endhead

    \DATAvalidation

\caption{Donnée de validation pour le composé \COMPOUNDname}
\label{tab:##}
\end{longtable}



\newpage



\paragraph{Modèle}
\subparagraph{}

\GRAPHcompound

\newpage