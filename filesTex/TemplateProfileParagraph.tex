\newpage

\subsubsection{\COMPOUNDname}

\paragraph{Summary}
\subparagraph{}
\begin{longtable}{l|l}
    \textbf{Parameter} & \textbf{Value} \\
    \hline\endhead
    LOD & \LODunitstype \\
    LOQ Min & \MINLOQunits \\
    LOQ Max & \MAXLOQunits \\
    \hline
    Correction & \CORRECTION \\
    Average Recovery & \AVERAGErecov \\
    Tolerance & \TOLERANCE \\
    Acceptance & \ACCEPT \\

\caption{Sommaire du composé \COMPOUNDname}
\label{tab:##}
\end{longtable}

\GRAPHprofile


\newpage


\paragraph{Trueness}
\subparagraph{}

\begin{longtable}{c|P{25mm}|P{25mm}|P{15mm}|P{15mm}|P{15mm}|P{20mm}|P{20mm}}
    \textbf{Level} & \textbf{Introduced Concentration} & \textbf{Calculated Concentration} & \textbf{Absolute Bias (\%)}
    & \textbf{Relative Bias (\%)} & \textbf{Recovery (\%)} & \textbf{Tolerance Absolute} & \textbf{Tolerance Relative}\\
    \hline\endhead

    \DATAtrueness

\caption{Justesse des mesures du composé \COMPOUNDname}
\label{tab:##}
\end{longtable}

\paragraph{Precision and Repeatability}
\subparagraph{}

\begin{longtable}{c|P{25mm}|P{23mm}|P{23mm}|P{24mm}|P{24mm}|P{12mm}}
    \textbf{Level} & \textbf{Introduced Concentration} & \textbf{Absolute Intermediate Precision} & \textbf{Relative Intermediate Precision}
    & \textbf{Absolute Repeatability} & \textbf{Relative Repeatability} & \textbf{Variance Ratio}\\
    \hline\endhead

    \DATAprecisionrepeat

\caption{Justesse des mesures du composé \COMPOUNDname}
\label{tab:##}
\end{longtable}

\paragraph{Validation Uncertainty}
\subparagraph{}

\begin{longtable}{c|P{25mm}|P{25mm}|P{25mm}|P{25mm}}
    \textbf{Level} & \textbf{Introduced Concentration} & \textbf{Calculated Concentration} & \textbf{Absolute Expanded Uncertainty}
    & \textbf{PC Expanded Uncertainty}\\
    \hline\endhead

    \DATAvaliduncertainty

\caption{Validation de l'incertitude des mesures du composé \COMPOUNDname}
\label{tab:##}
\end{longtable}

\GRAPHlinearity

\paragraph{Validation}
\subparagraph{}
Le tableau suivant présente les données de validation pour le composé \COMPOUNDname. Les données retrouvées sont celle avec le facteur de correction de XXXXXX.

\begin{longtable}{c|c|P{25mm}|P{20mm}|P{25mm}|P{20mm}|P{20mm}}
    \textbf{Série} & \textbf{Niveau} & \textbf{Concentration} & \textbf{Réponse} & \textbf{Concentration Calculé} & \textbf{Biais Absolu} & \textbf{Biais Relatif}\\
    \hline\endhead

    \DATAvalidation

\caption{Donnée de validation pour le composé \COMPOUNDname}
\label{tab:##}
\end{longtable}

\REGRESSIONinfo