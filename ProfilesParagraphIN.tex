\subsubsection{\COMPOUDname}

\paragraph{Données}
\subparagraph{Calibration}
Le tableau suivant présente les données de calibration pour le composé \COMPOUDname.
\begin{table}[h!]
    \centering
    \begin{tabular}{c|c|c|c}
        \textbf{Série} & \textbf{Niveau} & \textbf{Concentration} & \textbf{Réponse}\\
        \hline
        \DATAcalibration
    \end{tabular}
    \caption{Donnée de calibration pour le composé \COMPOUDname}
    \label{tab:##}
\end{table}
\subparagraph{Validation}
Le tableau suivant présente les données de validation pour le composé \COMPOUDname. Les données retrouvées sont celle avec le facteur de correction de XXXXXX.
\begin{table}[h!]
    \centering
    \begin{tabular}{c|c|c|c|c|c}
        \textbf{Série} & \textbf{Niveau} & \textbf{Concentration} & \textbf{Réponse} & \textbf{Retrouvé} & \textbf{Corrigé}  \\
        \hline
        \DATAvalidation
    \end{tabular}
    \caption{Donnée de validation pour le composé \COMPOUDname}
    \label{tab:##}
\end{table}
\newpage

\paragraph{Modèle}
\subparagraph{}
\GRAPHcompound